\documentclass{article}
\usepackage[spanish]{babel}
\usepackage[a4paper,
            top=2cm,
            bottom=2cm,
            left=3cm,
            right=3cm,
            headheight=36pt,
            nomarginpar,
            includehead,
            includefoot,
            ]{geometry}
\usepackage{graphicx}
\usepackage{parskip}
\usepackage{fancyhdr}
\usepackage{xcolor}
\usepackage{enumitem}
\usepackage[many, minted, listings]{tcolorbox}


% Opciones para entornos de código fuente
\definecolor{bg}{RGB}{29, 35, 49}
\definecolor{framecolor}{RGB}{125, 138, 151}

\newtcbinputlisting{\inputcode}[3][]{
    listing only,
    listing engine=minted,
    listing file={#3},
    minted language={#2},
    minted style=lightbulb,
    breakable,
    colback=bg,
    colframe=framecolor,
    #1
}


% Opciones para entornos de pseudocódigo
\lstset{
    breakatwhitespace=true,             % sets if automatic breaks should only happen at whitespace
    breaklines=true,                    % sets automatic line breaking
    keepspaces=true,                    % keeps spaces in text, useful for keeping indentation of code (possibly needs columns=flexible)
    columns=flexible,
    escapechar=|,
    literate=   {á}{{\'a}}1             % corrige errores de utf-8
                {é}{{\'e}}1
                {í}{{\'i}}1
                {ó}{{\'o}}1
                {ú}{{\'u}}1,
}


% Encabezado y pie de página
\fancyhf{}
\lhead{\includegraphics[height=32pt]{img/logo-uncuyo-fing.pdf}}
\rhead{ Licenciatura en Ciencias de la Computación \\
        Algoritmos y Estructuras de Datos II \\
        TP N\textsuperscript{o} 3: Árboles N-arios Trie}
\rfoot{\thepage}
\pagestyle{fancy}


\begin{document}
\input{titulo.tex}
\section*{Parte 1}

\textbf{Importante:} Los ejercicios de esta primera parte tienen como objetivo codificar las diferentes funciones básicas necesarias para implementar un Trie.

A partir de estructuras definidas como:

\begin{lstlisting}
|\textbf{class}| Trie:
    root = |\textbf{None}|

|\textbf{class}| TrieNode:
    parent = |\textbf{None}|
    children = |\textbf{None}|
    key = |\textbf{None}|
    isEndOfWord = |\textbf{None}|
\end{lstlisting}


\subsection*{Ejercicio 1}
Crear un módulo de nombre \textbf{trie.py} que implemente las siguientes especificaciones de las operaciones elementales para el \textbf{TAD Trie}.
\begin{lstlisting}
|\textbf{insert(T, element)}|
    |\textbf{Descripción}|: insert un elemento en T, siendo T un Trie.
    |\textbf{Entrada}|: El Trie sobre la cual se quiere agregar el elemento (Trie) y el valor del elemento (palabra) a agregar.
    |\textbf{Salida}|: No hay salida definida.

|\textbf{search(T, element)}|
    |\textbf{Descripción}|: Verifica que un elemento se encuentre dentro del Trie.
    |\textbf{Entrada}|: El Trie sobre la cual se quiere buscar el elemento (Trie) y el valor del elemento (palabra).
    |\textbf{Salida}|: Devuelve False o True según se encuentre el elemento.
\end{lstlisting}
\subsubsection*{Solución}
\inputcode{python3}{./code/snippets/ejercicio1a.py}
\inputcode{python3}{./code/snippets/ejercicio1b.py}


\subsection*{Ejercicio 2 (no code)}
Sabiendo que el orden de complejidad para el peor caso de la operación search() es de $O(m |\Sigma|)$. Proponga una versión de la operación search() cuya complejidad sea $O(m)$.
\subsubsection*{Solución}
La implementación de search() en el ejercicio 1 ya es de orden O(m), ya que en cada nodo, no es necesario buscar al siguiente nodo hijo recorriendo una lista de largo $|\Sigma|$, sino que se accede a él directamente mediante acceso indexado en un arreglo de longitud $|\Sigma|$. Si se busca el nodo con la letra 'a', simplemente se accede al elemento 0 del arreglo. Si se quiere el nodo con la letra 'b', se accede al índice 1, y así para cualquier letra. Para calcular el índice correspondiente a cada letra, se resta 97 al valor asociado a la misma en ASCII (la letra debe ser minúsucula).


\subsection*{Ejercicio 3}
\begin{lstlisting}
|\textbf{delete(T, element)}|
    |\textbf{Descripción}|: Elimina un elemento se encuentre dentro del Trie.
    |\textbf{Entrada}|: El Trie sobre la cual se quiere eliminar el elemento (Trie) y el valor del elemento (palabra) a eliminar.
    |\textbf{Salida}|: Devuelve False o True según se haya eliminado el elemento.
\end{lstlisting}
\subsubsection*{Solución}
\inputcode{python3}{./code/snippets/ejercicio3.py}


\section*{Parte 2}
\subsection*{Ejercicio 4}
Implementar un algoritmo que dado un árbol Trie T, un patrón p (prefijo) y un entero n, escriba todas las palabras del árbol que empiezan por p y sean de longitud n.
\subsubsection*{Solución}
\inputcode{python3}{./code/snippets/ejercicio4.py}


\subsection*{Ejercicio 5}
Implementar un algoritmo que dado los Trie T1 y T2 devuelva True si estos pertenecen al mismo documento y False en caso contrario. Se considera que un Trie pertenece al mismo documento cuando:
\begin{enumerate}
    \item Ambos Trie sean iguales (esto se debe cumplir)
    \item Si la implementación está basada en LinkedList, considerar el caso donde las palabras hayan sido insertadas en un orden diferente.
\end{enumerate}
En otras palabras, analizar si todas las palabras de T1 se encuentran en T2.

Analizar el costo computacional.
\subsubsection*{Solución}
\inputcode{python3}{./code/snippets/ejercicio5.py}


\subsection*{Ejercicio 6}
Implemente un algoritmo que dado el Trie T devuelva True si existen en el documento T dos cadenas invertidas. Dos cadenas son invertidas si se leen de izquierda a derecha y contiene los mismos caracteres que si se lee de derecha a izquierda, ej: abcd y dcba son cadenas invertidas, gfdsa y asdfg son cadenas invertidas, sin embargo abcd y dcka no son invertidas ya que difieren en un carácter.
\subsubsection*{Solución}
% \inputcode{python3}{./code/snippets/ejercicio6.py}


\subsection*{Ejercicio 7}
Un corrector ortográfico interactivo utiliza un Trie para representar las palabras de su diccionario. Queremos añadir una función de auto-completar (al estilo de la tecla TAB en Linux): cuando estamos a medio escribir una palabra, si sólo existe una forma correcta de continuarla entonces debemos indicarlo.

Implementar la función autoCompletar(Trie, cadena) dentro del módulo trie.py, que dado el árbol Trie T y la cadena devuelve la forma de auto-completar la palabra. Por ejemplo, para la llamada autoCompletar(T, ‘groen’) devolvería “land”, ya que podemos tener “groenlandia” o “groenlandés” (en este ejemplo la palabra groenlandia y groenlandés pertenecen al documento que representa el Trie). Si hay varias formas o ninguna, devolvería la cadena vacía. Por ejemplo, autoCompletar(T, ma’) devolvería “” (cadena vacia) si T presenta las cadenas “madera” y “mama”
\subsubsection*{Solución}
% \inputcode{python3}{./code/snippets/ejercicio7.py}


\end{document}
