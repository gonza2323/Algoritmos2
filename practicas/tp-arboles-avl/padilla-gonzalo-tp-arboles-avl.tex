\documentclass{article}
\usepackage[spanish]{babel}
\usepackage[a4paper,
            top=2cm,
            bottom=2cm,
            left=3cm,
            right=3cm,
            headheight=36pt,
            nomarginpar,
            includehead,
            includefoot,
            ]{geometry}
\usepackage{graphicx}
\usepackage{parskip}
\usepackage{fancyhdr}
\usepackage{xcolor}
\usepackage{enumitem}
\usepackage[many, minted]{tcolorbox}
\usepackage{listings}


% Opciones para entornos de código fuente
\definecolor{bg}{RGB}{29, 35, 49}
\definecolor{framecolor}{RGB}{125, 138, 151}

\newtcbinputlisting{\inputcode}[3][]{
    listing only,
    listing engine=minted,
    listing file={#3},
    minted language={#2},
    minted style=lightbulb,
    breakable,
    colback=bg,
    colframe=framecolor,
    #1
}


% Opciones para entornos de pseudocódigo
\lstset{
    breakatwhitespace=true,             % sets if automatic breaks should only happen at whitespace
    breaklines=true,                    % sets automatic line breaking
    keepspaces=true,                    % keeps spaces in text, useful for keeping indentation of code (possibly needs columns=flexible)
    columns=flexible,
    escapechar=|,
    literate=   {á}{{\'a}}1             % corrige errores de utf-8
                {é}{{\'e}}1
                {í}{{\'i}}1
                {ó}{{\'o}}1
                {ú}{{\'u}}1,
}


% Encabezado y pie de página
\fancyhf{}
\lhead{\includegraphics[height=32pt]{img/logo-uncuyo-fing.pdf}}
\rhead{ Licenciatura en Ciencias de la Computación \\
        Algoritmos y Estructuras de Datos II \\
        TP N\textsuperscript{o} 2: Árboles Balanceados AVL}
\rfoot{\thepage}
\pagestyle{fancy}


\begin{document}
\input{titulo.tex}
\section*{Parte 1}

\textbf{Importante:} Los ejercicios de esta primera parte tienen como objetivo codificar las diferentes funciones básicas necesarias para implementar un árbol AVL.

A partir de estructuras definidas como:

\begin{lstlisting}
|\textbf{class}| AVLTree:
    root = |\textbf{None}|

|\textbf{class}| AVLNode:
    parent = |\textbf{None}|
    leftnode = |\textbf{None}|
    rightnode = |\textbf{None}|
    key = |\textbf{None}|
    value = |\textbf{None}|
    bf = |\textbf{None}|
\end{lstlisting}

Copiar y adaptar todas las operaciones del \textbf{binarytree.py} (i.e. insert(), delete(), search(), etc.) al nuevo módulo \textbf{avltree.py}. Notar que estos luego deberán ser implementados para cumplir que la propiedad de un árbol AVL
\subsubsection*{Solución}
Ver ejercicios 4 y 5.


\subsection*{Ejercicio 1}
Crear un modulo de nombre \textbf{avltree.py}. Implementar las siguientes funciones:
\begin{lstlisting}
|\textbf{rotateLeft(Tree, avlnode)}|
    |\textbf{Descripción}|: Implementa la operación rotación a la izquierda
    |\textbf{Entrada}|: Un Tree junto a un AVLnode sobre el cual se va a operar la rotación a la izquierda
    |\textbf{Salida}|: retorna la nueva raíz

|\textbf{rotateRight(Tree, avlnode)}|
    |\textbf{Descripción}|: Implementa la operación rotación a la derecha
    |\textbf{Entrada}|: Un Tree junto a un AVLnode sobre el cual se va a operar la rotación a la derecha
    |\textbf{Salida}|: retorna la nueva raíz
\end{lstlisting}
\pagebreak
\subsubsection*{Solución}
\inputcode{python3}{./code/snippets/ejercicio1.py}


\pagebreak
\subsection*{Ejercicio 2}
Implementar una función recursiva que calcule el elemento balanceFactor de cada subárbol siguiendo la siguiente especificación:
\begin{lstlisting}
|\textbf{calculateBalance(AVLTree)}|
    |\textbf{Descripción}|: Calcula el factor de balanceo de un árbol binario de búsqueda.
    |\textbf{Entrada}|: El árbol AVL sobre el cual se quiere operar.
    |\textbf{Salida}|: El árbol AVL con el valor de balanceFactor para cada subárbol
\end{lstlisting}
\subsubsection*{Solución}
\inputcode{python3}{./code/snippets/ejercicio2.py}


\pagebreak
\subsection*{Ejercicio 3}
Implementar una función en el modulo avltree.py de acuerdo a  las siguientes especificaciones:
\begin{lstlisting}
|\textbf{reBalance(AVLTree)}|
    |\textbf{Descripción}|: balancea un árbol binario de búsqueda. Para esto se deberá primero calcular el balanceFactor del árbol y luego en función de esto aplicar la estrategia de rotación que corresponda.
    |\textbf{Entrada}|: El árbol binario de tipo AVL  sobre el cual se quiere operar.
    |\textbf{Salida}|: Un árbol binario de búsqueda balanceado. Es decir luego de esta operación se cumple que la altura (h) de su subárbol derecho e izquierdo difieren a lo sumo en una unidad.
\end{lstlisting}
\subsubsection*{Solución}
\setlength{\fboxsep}{0pt}
\inputcode{python3}{./code/snippets/ejercicio3.py}


\pagebreak
\subsection*{Ejercicio 4}
Implementar la operación \textbf{insert()} en  el módulo \textbf{avltree.py} garantizando que el árbol  binario resultante sea un árbol AVL.
\subsubsection*{Solución}
\inputcode{python3}{./code/snippets/ejercicio4.py}


\pagebreak
\subsection*{Ejercicio 5}
Implementar la operación \textbf{delete()} en  el módulo \textbf{avltree.py} garantizando que el árbol  binario resultante sea un árbol AVL.
\subsubsection*{Solución}
\inputcode{python3}{./code/snippets/ejercicio5.py}


\pagebreak
\section*{Parte 2}
\subsection*{Ejercicio 6}
Responder V o F y justificar su respuesta:
\begin{enumerate}[label=\alph*)]
\item \underline{ F } En un AVL el penúltimo nivel tiene que estar completo

El siguiente contraejemplo demuestra que la proposición es falsa.
\begin{center}
    \includegraphics[scale=0.7]{./img/ej6a-arbol.pdf}
\end{center}

Como puede verse, es un AVL ya que el balance factor de cada nodo es -1, 0 o 1, pero su penúltimo nivel está incompleto.


\item \underline{ V } Un AVL donde todos los nodos tengan factor de balance 0 es completo

Un árbol es completo si todos los nodos de todos sus niveles, excepto el último, tienen 2 hijos. Como todos los nodos del árbol tienen balance factor 0, entonces cada uno debe tener 0 hijos, o tener 2 hijos que formen subárboles de la misma altura.

Sea $h$ la altura del árbol, analicemos cada nivel $l$. En el primer nivel $(l=0)$, si $h=l=0$, entonces la raíz deberá tener necesariamente 0 hijos. En tal caso, se tiene un árbol completo, ya que los nodos de su último (y único) nivel tienen 0 hijos. En cambio, si $l\neq h$, entonces necesariamente deberá tener algún hijo. Pero como ya vimos, solo puede tener 0 o 2 hijos, por lo que debe tener 2. Además, los subárboles que formen sus hijos deberán tener ambos la misma altura $h-l-1$.

Ahora analizamos el siguiente nivel $(l=1)$. Se tienen dos posibles casos.
\begin{enumerate}[label=\roman{*})]
    \item $(l=h)$ En tal caso, estamos en el último nivel, y por lo tanto cada nodo del mismo debe tener 0 hijos.
    \item $(l<h)$ En este caso, algún nodo debe tener 2 hijos. Pero como dijimos que cada subárbol de este nivel debe tener la misma altura $(h-l)$, entonces \textit{todos} los nodos de este nivel tienen 2 hijos. Como cada nodo tiene balance factor 0, entonces sus hijos deben también formar subárboles de la misma altura $h-l-1$.
\end{enumerate}

Y así para cada $l = 1, 2, \dots, h$ Entonces, puede verse que para todos los niveles, excepto el último, cada nodo tiene 2 hijos, entonces se tiene un árbol completo.


\item \underline{ F } En la inserción en un AVL, si al actualizarle el factor de balance al padre del nodo insertado éste no se desbalanceó, entonces no hay que seguir verificando hacia arriba porque no hay cambios en los factores de balance.

Se demuestra mediante el siguiente contraejemplo

\begin{center}
    \includegraphics[scale=0.4]{./img/ej6c.pdf}
\end{center}

Al insertar un nodo a la izquierda del árbol, su padre no se desbalanceó (es -1, 0 o 1). Aún así, cambiaron los balance factor de todos los nodos superiores.


\item \underline{ V } En todo AVL existe al menos un nodo con factor de balance 0.

Todas las hojas tienen balance factor 0, por lo que es verdadero. Si no se toman en cuenta las mismas, entonces la proposición es falsa. Basta tomar un árbol que solo tiene dos nodos: la raíz y un solo hijo. En este caso el hijo es un nodo hoja con balance factor 0, y la raíz, que es el único nodo interior, tiene balance factor 1, entonces el árbol no tiene ningún nodo interior con balance factor 0.

\end{enumerate}


\subsection*{Ejercicio 7}
Sean $A$ y $B$ dos AVL de $m$ y $n$ nodos respectivamente y sea $x$ un key cualquiera de forma tal que para todo key $a\in A$ y para todo key $b\in B$ se cumple que $a < x < b$. Plantear un algoritmo $O(\log n + \log m)$ que devuelva un AVL que contenga los key de $A$, el key $x$ y los key de $B$.
\subsubsection*{Solución}
Se calculan las alturas $h_A$ y $h_B$ de ambos árboles (operaciones de orden $O(\log m)$ y $O(\log n)$, respectivamente). Si $|h_A-h_B|<2$ entonces simplemente basta conectar el árbol $A$ como hijo izquierdo de $x$, y a $B$ como su hijo derecho, y se tendrá un árbol AVL con $x$ como su raíz. En cambio, si $|h_A-h_B|\geq 2$, entonces insertaremos el subárbol de menor altura en el de mayor altura de la siguiente manera:
\begin{enumerate}
    \item Recorremos el árbol de mayor altura por su derecha si se trata de $A$, o por su izquierda si es $B$, hasta haber recorrido una distancia igual a la diferencia entre la altura del árbol más alto y la altura del más corto. Es decir, una vez que nos detenemos, estaremos en la raíz de un subárbol con altura igual a la del árbol más corto.
    \item A tal nodo, lo reemplazamos por $x$, colocando al mismo como hijo de $x$. Si estamos en $A$, como hijo izquierdo, si no, como hijo derecho. Esto preserva el orden de todas las llaves, pero ahora $x$ es un nodo muy desbalanceado.
    \item Finalmente, colocamos al árbol más corto como el hijo restante de $x$. Esto balancea al árbol, ya que debajo de $x$, el subárbol que ahora es hermano del árbol más corto, tenía una altura igual a él, ya que recorrimos el árbol originalmente hasta esa distancia. También se preserva el orden de las keys.
\end{enumerate}


\pagebreak
\subsection*{Ejercicio 8}
Considere una rama truncada en un AVL como un camino simple desde la raíz hacia un nodo que tenga una referencia None (que le falte algún hijo). Demuestre que la mínima longitud (cantidad de aristas) que puede tener una rama truncada en un AVL de altura $h$ es $h/2$ (tomando la parte entera por abajo).

Cualquier camino desde la raíz hasta un nodo que no esté completo puede ser una rama truncada según la definición del ejercicio. Dicho nodo puede no ser necesariamente un nodo hoja.
\subsubsection*{Solución}
Encontrar la longitud mínima que puede tener una rama truncada en un AVL de altura $h$ es equivalente a encontrar la mínima cantidad de niveles completos del mismo. Esto es cierto ya que si el siguiente nivel está completo, entonces todos los nodos del nivel actual tienen 2 hijos (entonces ninguno es el final de una rama truncada), y si el nivel actual no es completo, entonces existe uno con menos de 2 hijos en el nivel anterior (entonces existe una rama truncada de menor longitud al nivel actual).

Podemos encontrar el nivel mínimo completo de un árbol siguiendo los recorridos de menor longitud posible. Si el árbol es de altura $h$, entonces existe un recorrido desde la raíz hasta algún nodo del último nivel, que contiene $h$ aristas y $h+1$ nodos. Para mantener el balance de la raíz, debe existir otro recorrido, que pase por el otro hijo de la raíz, y que tenga altura $h$ o $h-1$. Como nos interesa encontrar el recorrido de menor longitud posible, nos concentramos en el caso en el que tiene longitud $h-1$.

De igual forma, desde la perspectiva del hijo de la raíz contenido en este segundo recorrido (en el nivel $n=1$), tal recorrido tiene un altura $h-2$ (comenzando desde el nivel 1), por lo que también debe ser el comienzo de otro recorrido de $h-3$ aristas, que pase por su otro hijo, para mantener su propio balance. En general, en este mínimo recorrido, el nodo del nivel $l$ forma parte de un recorrido que comienza en él, con altura $h-2l$ a partir de si mismo, y debe a su vez ser el comienzo de un recorrido de altura $h-2l-1$.

Nótese que ambos números pueden ser cero o negativos. Cuando el primero sea cero, significa que el nodo de ese nivel es el final del recorrido, y no existe recorrido menor, entonces estamos en el mínimo nivel completo, y el nodo no tiene hijos (es una hoja). Si, en cambio, el segundo número es 0, y el primero positivo, quiere decir que para mantener al nodo balanceado, uno de sus hijos puede no existir (pero no ninguno), entonces hemos encontrado también el mínimo nivel completo y, por lo tanto, la longitud mínima de una rama truncada.

Finalmente, tenemos que $h-2l=0$ si $l=\frac{h}{2}$, y que $h-2l-1=0$ si $l=\frac{h-1}{2}$. Si $h$ es par, solo la primera ecuación tiene solución entera, por lo que utilizamos la fórmula correspondiente. De la misma forma, si $h$ es impar, solo la segunda tiene solución entera. La fórmula $l = \lfloor h/2 \rfloor$ logra que alguno de los dos números sea 0, y por lo tanto indica la altura mínima de una rama truncada, que es lo que se quería probar.


\pagebreak
\section*{Parte 3}
\subsection*{Ejercicios Opcionales}
\begin{enumerate}
    \item Si n es la cantidad de nodos en un árbol AVL, implemente la operación \textbf{height()} en el módulo \textbf{avltree.py} que determine su altura en $O(\log n)$. Justifique el por qué de dicho orden.
    \subsubsection*{Solución}
    \inputcode{python3}{./code/snippets/ejercicioopcional1.py}
    
    \item Considere una modificación en el módulo \textbf{avltree.py} donde a cada nodo se le ha agregado el campo \textbf{count} que almacena el número de nodos que hay en el subárbol en el que él es raíz. Programe un algoritmo $O(\log n)$ que determine la cantidad de nodos en el árbol cuyo valor del key se encuentra en un intervalo $[a, b]$ dado como parámetro. Explique brevemente por qué el algoritmo programado por usted tiene dicho orden.
\end{enumerate}


\end{document}
