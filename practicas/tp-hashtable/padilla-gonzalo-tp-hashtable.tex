\documentclass{article}
\usepackage[spanish]{babel}
\usepackage[a4paper,
            top=2cm,
            bottom=2cm,
            left=3cm,
            right=3cm,
            headheight=36pt,
            nomarginpar,
            includehead,
            includefoot,
            ]{geometry}
\usepackage{graphicx}
\usepackage{parskip}
\usepackage{fancyhdr}
\usepackage{xcolor}
\usepackage{enumitem}
\usepackage[many, minted, listings]{tcolorbox}
\usepackage{multicol}


% Opciones para entornos de código fuente
\definecolor{bg}{RGB}{29, 35, 49}
\definecolor{framecolor}{RGB}{125, 138, 151}

\newtcbinputlisting{\inputcode}[3][]{
    listing only,
    listing engine=minted,
    listing file={#3},
    minted language={#2},
    minted style=lightbulb,
    breakable,
    colback=bg,
    colframe=framecolor,
    #1
}


% Opciones para entornos de pseudocódigo
\lstset{
    breakatwhitespace=true,             % sets if automatic breaks should only happen at whitespace
    breaklines=true,                    % sets automatic line breaking
    keepspaces=true,                    % keeps spaces in text, useful for keeping indentation of code (possibly needs columns=flexible)
    columns=flexible,
    escapechar=|,
    literate=   {á}{{\'a}}1             % corrige errores de utf-8
                {é}{{\'e}}1
                {í}{{\'i}}1
                {ó}{{\'o}}1
                {ú}{{\'u}}1,
}


% Encabezado y pie de página
\fancyhf{}
\lhead{\includegraphics[height=32pt]{img/logo-uncuyo-fing.pdf}}
\rhead{ Licenciatura en Ciencias de la Computación \\
        Algoritmos y Estructuras de Datos II \\
        TP N\textsuperscript{o} 4: Hash Tables}
\rfoot{\thepage}
\pagestyle{fancy}


\begin{document}
\input{titulo.tex}

\section*{Parte 1}
\subsection*{Ejercicio 1}
Ejemplificar que pasa cuando insertamos las llaves 5, 28, 19, 15, 20, 33, 12, 17, 10 en un HashTable con la colisión resulta por el método de chaining. Permita que la tabla tenga 9 slots y la función de hash:
\begin{equation}
    H(k) = k \bmod 9
\end{equation}
\subsubsection*{Solución}
Para cada llave se obtienen las siguientes posiciones en la tabla a partir de la función hash:
\begin{multicols}{2}
\begin{enumerate}[topsep=0pt, parsep=0pt, itemsep=4pt]
    \item $H(5)  = 5$
    \item $H(28) = 1$
    \item $H(19) = 1$ (Colisión)
    \item $H(15) = 6$
    \item $H(20) = 2$
    \item $H(33) = 6$ (Colisión)
    \item $H(12) = 3$
    \item $H(17) = 8$
    \item $H(10) = 1$ (Colisión)
\end{enumerate}
\end{multicols}
Como utilizamos el método de chaining, cuando se produce una colisión, encadenamos la key a la última key en esa posición, utilizando una lista enlazada. La tabla resultante quedaría:

\begin{center}
\begin{tabular}{|c|l|}
    \hline
    i & Keys \\ \hline
    0 &      \\ \hline
    1 & 28 $\rightarrow$ 19 $\rightarrow$ 10 \\ \hline
    2 & 20   \\ \hline
    3 & 12   \\ \hline
    4 &      \\ \hline
    5 & 5    \\ \hline
    6 & 15 $\rightarrow$ 33 \\ \hline
    7 &      \\ \hline
    8 & 17   \\ \hline
\end{tabular}
\end{center}


\subsection*{Ejercicio 2}
A partir de una definición de diccionario como la siguiente:

dictionary = Array(m,0) \# una sugerencia de implementación, se puede usar una lista de python.

Crear un módulo de nombre dictionary.py que implemente las siguientes especificaciones de las operaciones elementales para el TAD diccionario.

Nota: dictionary puede ser redefinido para lidiar con las colisiones por encadenamiento.

\begin{lstlisting}
|\textbf{insert(D, key, value)}|
|\qquad \textbf{Descripción}|: Inserta un key en una posición determinada por la función de hash (1) en el diccionario (dictionary). Resolver colisiones por encadenamiento. En caso de keys duplicados se anexan a la lista.
|\qquad \textbf{Entrada}|: el diccionario sobre el cual se quiere realizar la inserción y el valor del key a insertar.        
|\qquad \textbf{Salida}|: Devuelve D.
\end{lstlisting}
\pagebreak
\begin{lstlisting}
|\textbf{search(D, key)}|
|\qquad \textbf{Descripción}|: Busca un key en el diccionario.
|\qquad \textbf{Entrada}|: El diccionario sobre el cual se quiere realizar la búsqueda (dictionary) y el valor del key a buscar.
|\qquad \textbf{Salida}|: Devuelve el value de la key. Devuelve None si el key no se encuentra.        

|\textbf{delete(D, key)}|
|\qquad \textbf{Descripción}|: Elimina un key en la posición determinada por la función de hash (1) del diccionario (dictionary).
|\qquad \textbf{Poscondición}|: Se debe marcar como None el key a eliminar.
|\qquad \textbf{Entrada}|: El diccionario sobre el se quiere realizar la eliminación y el valor del key que se va a eliminar.
|\qquad \textbf{Salida}|: Devuelve D.
\end{lstlisting}
\subsubsection*{Solución}
\inputcode{python3}{./code/snippets/ejercicio1a.py}
\inputcode{python3}{./code/snippets/ejercicio1b.py}
\pagebreak
\inputcode{python3}{./code/snippets/ejercicio1c.py}


\section*{Parte 2}
\subsection*{Ejercicio 3}
Considerar una tabla hash de tamaño $m = 1000$ y una función de hash correspondiente al método de la multiplicación donde $A = (\sqrt{5}-1)/2)$. Calcular las ubicaciones para las claves 61, 62, 63, 64 y 65.
\subsubsection*{Solución}
Se obtienen las siguientes ubicaciones:
\begin{itemize}
    \item $H(61) = \lfloor m\ (61\ (\sqrt{5}-1)/2 \bmod 1) \rfloor = 700$
    \item $H(62) = \lfloor m\ (62\ (\sqrt{5}-1)/2 \bmod 1) \rfloor = 318$
    \item $H(63) = \lfloor m\ (63\ (\sqrt{5}-1)/2 \bmod 1) \rfloor = 936$
    \item $H(64) = \lfloor m\ (64\ (\sqrt{5}-1)/2 \bmod 1) \rfloor = 554$
    \item $H(65) = \lfloor m\ (65\ (\sqrt{5}-1)/2 \bmod 1) \rfloor = 172$
\end{itemize}


\pagebreak
\subsection*{Ejercicio 4}
Implemente un algoritmo lo más eficiente posible que devuelva True o False a la siguiente proposición: dado dos strings $s_1 \dots s_k$ y $p_1 \dots p_k$, se quiere encontrar si los caracteres de $p_1 \dots p_k$ corresponden a una permutación de $s_1 \dots s_k$. Justificar el costo en tiempo de la solución propuesta.
\subsubsection*{Solución}
\inputcode{python3}{./code/snippets/ejercicio4.py}


\pagebreak
\subsection*{Ejercicio 5}
Implemente un algoritmo que devuelva True si la lista que recibe de entrada tiene todos sus elementos únicos, y Falso en caso contrario. Justificar el costo en tiempo de la solución propuesta.
\subsubsection*{Solución}
\inputcode{python3}{./code/snippets/ejercicio5.py}


\subsection*{Ejercicio 6}
Los nuevos códigos postales argentinos tienen la forma $cddddccc$, donde $c$ indica un carácter (A - Z) y $d$ indica un dígito $0, ... , 9$. Por ejemplo, C1024CWN es el código postal que representa a la calle XXXX a la altura 1024 en la Ciudad de Mendoza. Encontrar e implementar una función de hash apropiada para los códigos postales argentinos.
\subsubsection*{Solución}
\inputcode{python3}{./code/snippets/ejercicio6.py}


\pagebreak
\subsection*{Ejercicio 7}
Implemente un algoritmo para realizar la compresión básica de cadenas utilizando el recuento de caracteres repetidos. Por ejemplo, la cadena `aabcccccaaa' se convertiría en `a2blc5a3'. Si la cadena ``comprimida'' no se vuelve más pequeña que la cadena original, su método debería devolver la cadena original. Puedes asumir que la cadena sólo tiene letras mayúsculas y minúsculas (a - z, A - Z). Justificar el costo en tiempo de la solución propuesta.
\subsubsection*{Solución}
\inputcode{python3}{./code/snippets/ejercicio7.py}


\subsection*{Ejercicio 8}
Se requiere encontrar la primera ocurrencia de un string $p_1 \dots p_k$ en uno más largo $a_1 \dots a_L$. Implementar esta estrategia de la forma más eficiente posible con un costo computacional menor a $O(K\cdot L)$ (solución por fuerza bruta). Justificar el coste en tiempo de la solución propuesta.
\subsubsection*{Solución}
\pagebreak
\inputcode{python3}{./code/snippets/ejercicio8.py}


\subsection*{Ejercicio 9}
Considerar los conjuntos de enteros $S = {s_1, \dots , s_n}$ y $T = {t_1, \dots , t_m}$. Implemente un algoritmo que utilice una tabla de hash para determinar si $S \subseteq T$ (S subconjunto de T). ¿Cuál es la complejidad temporal del caso promedio del algoritmo propuesto?
\subsubsection*{Solución}
\inputcode{python3}{./code/snippets/ejercicio9.py}


\section*{Parte 3}
\subsection*{Ejercicio 10}
Considerar la inserción de las siguientes llaves: 10; 22; 31; 4; 15; 28; 17; 88; 59 en una tabla hash de longitud $m = 11$ utilizando direccionamiento abierto con una función de hash $h'(k) = k$. Mostrar el resultado de insertar estas llaves utilizando:
\begin{enumerate}
    \item Linear probing.
    \item Quadratic probing con $c_1 = 1$ y $c_2 = 3$.
    \item Double hashing con $h_1(k) = k$ y $h_2(k) = 1 +(k \bmod ( m - 1))$.
\end{enumerate}
\subsubsection*{Solución}
\begin{center}
\begin{tabular}{|c|l|l|l|}
    \hline
    i  & Linear & Quadratic & Double Hashing \\ \hline
    0  & 22     & 22        & 22             \\ \hline
    1  & 88     &           &                \\ \hline
    2  &        & 88        & 59             \\ \hline
    3  &        & 17        & 17             \\ \hline
    4  & 4      & 4         & 4              \\ \hline
    5  & 15     &           & 15             \\ \hline
    6  & 28     & 28        & 28             \\ \hline
    7  & 17     & 59        & 88             \\ \hline
    8  & 59     & 15        &                \\ \hline
    9  & 31     & 31        & 31             \\ \hline
    10 & 10     & 10        & 10             \\ \hline
\end{tabular}
\end{center}


\subsection*{Ejercicio 11 (opcional)}
Implementar las operaciones de insert() y delete() dentro de una tabla hash vinculando todos los nodos libres en una lista. Se asume que un slot de la tabla puede almacenar un indicador (flag), un valor, junto a una o dos referencias (punteros). Todas las operaciones de diccionario y manejo de la lista enlazada deben ejecutarse en O(1). La lista debe estar doblemente enlazada o con una simplemente enlazada alcanza?


\subsection*{Ejercicio 12}
Las llaves 12, 18, 13, 2, 3, 23, 5 y 15 se insertan en una tabla hash inicialmente vacía de longitud 10 utilizando direccionamiento abierto con función hash $h(k) = k \bmod 10$ y exploración lineal (linear probing). ¿Cuál es la tabla hash resultante? Justifique.

\begin{center}
    \includegraphics*[width=340px]{./img/ej12.png}
\end{center}
\subsubsection*{Solución}
Para cada llave se obtienen las siguientes posiciones en la tabla a partir de la función hash:
\begin{multicols}{2}
\begin{enumerate}[topsep=0pt, parsep=0pt, itemsep=2pt]
    \item $h(12, 0) = 2$
    \item $h(18, 0) = 8$
    \item $h(13, 0) = 3$
    \item $h(2 , 0) = 2$ (Colisión)
    \item $h(2 , 1) = 3$ (Colisión)
    \item $h(2 , 2) = 4$
    \item $h(3 , 0) = 3$ (Colisión)
    \item $h(3 , 1) = 4$ (Colisión)
    \item $h(3 , 2) = 5$
    \item $h(23, 0) = 3$ (Colisión)
    \item $h(23, 1) = 4$ (Colisión)
    \item $h(23, 2) = 5$ (Colisión)
    \item $h(23, 3) = 6$
    \item $h(5 , 0) = 5$ (Colisión)
    \item $h(5 , 1) = 6$ (Colisión)
    \item $h(5 , 2) = 7$
    \item $h(15, 0) = 5$ (Colisión)
    \item $h(15, 1) = 6$ (Colisión)
    \item $h(15, 2) = 7$ (Colisión)
    \item $h(15, 3) = 8$ (Colisión)
    \item $h(15, 4) = 9$
\end{enumerate}
\end{multicols}
Lo cual coincide con la tabla (C).


\subsection*{Ejercicio 13}
Una tabla hash de longitud 10 utiliza direccionamiento abierto con función hash $h(k)=k \bmod 10$, y exploración lineal (linear probing). Después de insertar 6 valores en una tabla hash vacía, la tabla es como se muestra a continuación.

\begin{center}
\begin{tabular}{|c|c|}
    \hline
    0 &    \\ \hline
    1 &    \\ \hline
    2 & 42 \\ \hline
    3 & 23 \\ \hline
    4 & 34 \\ \hline
    5 & 52 \\ \hline
    6 & 46 \\ \hline
    7 & 33 \\ \hline
    8 &    \\ \hline
    9 &    \\ \hline
\end{tabular}
\end{center}

¿Cuál de las siguientes opciones da un posible orden en el que las llaves podrían haber sido
insertadas en la tabla? Justifique
\begin{enumerate}[label=(\Alph*), ref=\Alph*]
    \item 46, 42, 34, 52, 23, 33
    \item 34, 42, 23, 52, 33, 46
    \item 46, 34, 42, 23, 52, 33
    \item 42, 46, 33, 23, 34, 52
\end{enumerate}
\subsubsection*{Solución}
\begin{enumerate}[label=(\Alph*), ref=\Alph*]
    \item No es un orden posible, ya que el 52, con $h(52) = 2$, debería ocupar la posición 3 (hasta ahora libre), al estar ocupada la posición 2 por el 42 insertado anteriormente.
    \item No es un orden posible, ya que el 33, con $h(33) = 3$, debería ocupar la posición 6 (hasta ahora libre), al estar ocupada la posición 3 por el 23, y las siguientes por 34 y 52.
    \item Es un orden posible, ya que las únicas colisiones son producidas por el 52, con $h(52) = 2$, que termina ocupando la posición 5 al estar ocupadas desde las posiciones desde la 2 hasta la 4; y el 33, con $h(33) = 3$, que termina ocupando la posición 7 al estar ocupadas las posiciones desde la 3 hasta la 6.
    \item No es un orden posible, ya que el 33, con $h(33) = 3$, termina ocupando la posición 3, ya que no está ocupada, en lugar de la 7.
\end{enumerate}
\end{document}
