\documentclass{article}
\usepackage[spanish]{babel}
\usepackage[a4paper,
            top=2cm,
            bottom=2cm,
            left=3cm,
            right=3cm,
            headheight=36pt,
            nomarginpar,
            includehead,
            includefoot,
            ]{geometry}
\usepackage{graphicx}
\usepackage{parskip}
\usepackage{fancyhdr}
\usepackage{xcolor}
\usepackage{enumitem}
\usepackage[many, minted, listings]{tcolorbox}


% Opciones para entornos de código fuente
\definecolor{bg}{RGB}{29, 35, 49}
\definecolor{framecolor}{RGB}{125, 138, 151}

\newtcbinputlisting{\inputcode}[3][]{
    listing only,
    listing engine=minted,
    listing file={#3},
    minted language={#2},
    minted style=lightbulb,
    breakable,
    colback=bg,
    colframe=framecolor,
    #1
}


% Opciones para entornos de pseudocódigo
\lstset{
    breakatwhitespace=true,             % sets if automatic breaks should only happen at whitespace
    breaklines=true,                    % sets automatic line breaking
    keepspaces=true,                    % keeps spaces in text, useful for keeping indentation of code (possibly needs columns=flexible)
    columns=flexible,
    escapechar=|,
    literate=   {á}{{\'a}}1             % corrige errores de utf-8
                {é}{{\'e}}1
                {í}{{\'i}}1
                {ó}{{\'o}}1
                {ú}{{\'u}}1,
}


% Encabezado y pie de página
\fancyhf{}
\lhead{\includegraphics[height=32pt]{img/logo-uncuyo-fing.pdf}}
\rhead{ Licenciatura en Ciencias de la Computación \\
        Algoritmos y Estructuras de Datos II \\
        TP N\textsuperscript{o} 4: Hash Tables}
\rfoot{\thepage}
\pagestyle{fancy}


\begin{document}
\input{titulo.tex}


\subsection*{Ejercicio 1}
Crear un módulo de nombre \textbf{trie.py} que implemente las siguientes especificaciones de las operaciones elementales para el \textbf{TAD Trie}.
\begin{lstlisting}
|\textbf{insert(T, element)}|
    |\textbf{Descripción}|: insert un elemento en T, siendo T un Trie.
    |\textbf{Entrada}|: El Trie sobre la cual se quiere agregar el elemento (Trie) y el valor del elemento (palabra) a agregar.
    |\textbf{Salida}|: No hay salida definida.

|\textbf{search(T, element)}|
    |\textbf{Descripción}|: Verifica que un elemento se encuentre dentro del Trie.
    |\textbf{Entrada}|: El Trie sobre la cual se quiere buscar el elemento (Trie) y el valor del elemento (palabra).
    |\textbf{Salida}|: Devuelve False o True según se encuentre el elemento.
\end{lstlisting}


\end{document}
